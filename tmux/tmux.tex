%%%%%%%%%%%%%%%%%%%%%%%%%%%%%%%%%%%%%%%%%
% Cheatsheet
% LaTeX Template
% Version 1.0 (12/12/15)
%
% This template has been downloaded from:
% http://www.LaTeXTemplates.com
%
% Original author:
% Michael Müller (https://github.com/cmichi/latex-template-collection) with
% extensive modifications by Vel (vel@LaTeXTemplates.com)
%
% License:
% The MIT License (see included LICENSE file)
%
% Modified by:
% parijatmishra
%%%%%%%%%%%%%%%%%%%%%%%%%%%%%%%%%%%%%%%%%

%----------------------------------------------------------------------------------------
%	PACKAGES AND OTHER DOCUMENT CONFIGURATIONS
%----------------------------------------------------------------------------------------

\documentclass[11pt]{scrartcl} % 11pt font size

\usepackage[utf8]{inputenc} % Required for inputting international characters
\usepackage[T1]{fontenc} % Output font encoding for international characters

\usepackage[margin=0pt, landscape]{geometry} % Page margins and orientation

\usepackage{graphicx} % Required for including images

\usepackage{color} % Required for color customization
\definecolor{mygray}{gray}{.75} % Custom color

\usepackage{url} % Required for the \url command to easily display URLs

\usepackage[ % This block contains information used to annotate the PDF
colorlinks=false, 
pdftitle={Cheatsheets}, 
pdfauthor={Parijat Mishra}, 
pdfsubject={Compilation of useful shortcuts}, 
pdfkeywords={Cheatsheet, tmux}
]{hyperref}

\setlength{\unitlength}{1mm} % Set the length that numerical units are measured in
\setlength{\parindent}{0pt} % Stop paragraph indentation

\renewcommand{\dots}{\ \dotfill{}\ } % Fills in the right amount of dots

\newcommand{\mytext}[1]{\textsf{\small #1}}
\newcommand{\command}[2]{#1~\dotfill{}~\mytext{#2}\\} % Custom command for adding a shorcut

\newcommand{\sectiontitle}[1]{\paragraph{#1} \ \\} % Custom command for subsection titles


\newcommand{\Kprefix}{\P\ }
\newcommand{\Kctrl}{$\land$}
\newcommand{\Kenter}{$\hookleftarrow$}
\newcommand{\Kalt}{$^{alt}$}
%----------------------------------------------------------------------------------------

\begin{document}

\begin{picture}(297,210) % Create a container for the page content

%----------------------------------------------------------------------------------------
%	TITLE SECTION 
%----------------------------------------------------------------------------------------

\put(10,200){% Position on the page to put the title
\fbox{\begin{minipage}[t]{85mm} % The size and alignment of the title
\section*{tmux} % Title
\end{minipage}}
}

%----------------------------------------------------------------------------------------
%	LEGEND SECTION
%----------------------------------------------------------------------------------------

\put(105,200){% Position on the page to put the legend
\fbox{\begin{minipage}[t]{180mm} % The size and alignment of the legend
\Kprefix \mytext{prefix}
\Kctrl\ \mytext{control}
\Kenter\ \mytext{enter}
\Kalt\ \mytext{alt/option}
\end{minipage}}
}
%----------------------------------------------------------------------------------------
%	FIRST COLUMN SPECIFICATION
%----------------------------------------------------------------------------------------

\put(10,180){% Divide the page
\fbox{\begin{minipage}[t]{85mm} % Create a box to house text

%----------------------------------------------------------------------------------------
%	SESSIONS
%----------------------------------------------------------------------------------------

\sectiontitle{Sessions}

\command{\$ tmux ls}{List}
\command{\$ tmux}{Create new, numbered}
\command{\$ tmux new -s \textit{name}}{Create new, named}
\command{\$ tmux new -A -t \textsl{name}}{Attach to or create session named \textsl{name}}
\command{\$ tmux a -t \textsl{n}}{Attach to session numbered or named \textsl{n}}
\command{\$ tmux a}{Attach to first session found}
\command{\$ tmux kill-session }{Kill all}
\command{\$ tmux kill-session -t \textsl{n}}{Kill session with name or number \textsl{n}}
\command{\Kprefix d}{Detach from current}

%----------------------------------------------------------------------------------------
%	WINDOWS
%----------------------------------------------------------------------------------------

\sectiontitle{Windows}

\command{\Kprefix c}{Create}
\command{\Kprefix \&}{Kill}
\command{\Kprefix w}{List}
\command{\Kprefix $n$}{Go to window $n$ ($n$: 0..9)}
\command{\Kprefix ,}{Name}

%----------------------------------------------------------------------------------------
%	PANES
%----------------------------------------------------------------------------------------

\sectiontitle{Panes}

\command{\Kprefix \%}{Split vertically}
\command{\Kprefix ''}{Split horizontally}
\command{\Kprefix x}{Kill}
\command{\Kprefix q, $n$}{Show pane numbers, select pane $n$}
\command{\Kprefix \{}{Move pane leftwards}
\command{\Kprefix \}}{Move pane rightwards}
\command{\Kprefix z}{Zoom current pane in/out}

\sectiontitle{Window $\leftrightarrow$ Pane}
\command{\Kprefix !}{Move current pane to new window}
\command{\Kprefix :join-pane -t $W.n$}{\\Move current pane to pane $W.n$, splitting it}
\command{\Kprefix :join-pane -s $W.n$}{\\Move pane $W.n$ to current pane, splitting it}

%----------------------------------------------------------------------------------------

\end{minipage}} % End the first column of text
} % End the first division of the page

%----------------------------------------------------------------------------------------
%	SECOND COLUMN SPECIFICATION 
%----------------------------------------------------------------------------------------

\put(105,180){ % Divide the page
\fbox{\begin{minipage}[t]{85mm} % Create a box to house text

%----------------------------------------------------------------------------------------
%	COPY MODE
%----------------------------------------------------------------------------------------

\sectiontitle{Copy mode (vi keys)}

\command{\Kprefix [}{Enter copy mode}
\command{\Kprefix ]}{Paste selection}

\mytext{Selection in copy mode:}\\
\command{<space>}{Start selection}
\command{v}{Toggle rectangle (vertical) mode}
\command{\Kenter}{Copy selection to clipboard and exit}
\command{q}{Cancel selection and exit}
\command{ESC}{Cancel selection, don't exit}

\mytext{Navigation in copy mode:}\\
\command{h|j|k|l}{Cursor left|down|up|right}
\command{b}{Previous word}
\command{w}{Next word start}
\command{e}{Next word end}
\command{0}{Start of line}
\command{\$}{End of line}
\command{\Kctrl-y}{Scroll up}
\command{\Kctrl-e}{Scroll down}
\command{H}{Cursor to top visible line}
\command{L}{Cursor to last visible line}
\command{M}{Cursor to middle visible line}
\command{\Kctrl-b}{Page up}
\command\Kctrl-u{}{Halfpage up}
\command\Kctrl-d{}{Halfpage down}
\command{\Kctrl-f}{Page down}
\command{g}{First line in buffer/history}
\command{G}{Last line in buffer}

\mytext{Searching:}\\
\command{/}{Search downwards}
\command{?}{Search upwards}
\command{n}{Jump to next match}
\command{N}{Jump to next match backwards}

%----------------------------------------------------------------------------------------

\end{minipage}} % End the second column of text
} % End the second division of the page

%----------------------------------------------------------------------------------------
%	THIRD COLUMN SPECIFICATION 
%----------------------------------------------------------------------------------------

\put(200,180){ % Divide the page
\fbox{\begin{minipage}[t]{85mm} % Create a box to house tex

%----------------------------------------------------------------------------------------
%	Configuration
%----------------------------------------------------------------------------------------

\sectiontitle{Configuration}

\mytext{Configuration file: \texttt{~/.tmux.conf}}
\\
\mytext{Use vi keys in copy mode:}\\
\texttt{setw -g mode-key vi}
\\
\mytext{Switch panes using \Kalt-<arrow>:}\\
\texttt{%
bind -n M-left select-pane -L\\
bind -n M-right select-pane -R\\
bind -n M-Up select-pane -U\\
bind -n M-Down select-pane -D\\
}
\\
\mytext{Use \P-| and \P-\_ to split panes:}\\
\texttt{%
unbind `"`\\
unbind `\%`\\
bind | split-window -h\\
bind \_ split-window -v\\
}
%----------------------------------------------------------------------------------------
%	FOOTNOTE
%----------------------------------------------------------------------------------------

%\vspace{\baselineskip}
%\linethickness{0.5mm} % Thickness of the footer line
%{\color{mygray}\line(1,0){30}} % Print the line with a custom color

%\footnotesize{
%Created by Parijat Mishra, 2017\\ 
%\url{http://github.com/parijatmishra/cheatsheets}\\
%				
%Released under the MIT license.
%}

%----------------------------------------------------------------------------------------

\end{minipage}} % End the third column of text
} % End the third division of the page
\end{picture} % End the container for the entire page

%----------------------------------------------------------------------------------------

\end{document}
